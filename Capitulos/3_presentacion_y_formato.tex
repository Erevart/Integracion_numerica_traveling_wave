%---------------------------------------
\section{PRESENTACIÓN Y FORMATO DE DOCUMENTOS} \label{sec:presentacion}
%---------------------------------------
El TFG/M se presentará en soporte electrónico óptico (CD/DVD) y con formato del tipo Portable Document File (PDF), de forma que sea posible su lectura e impresión sin restricción alguna. El documento final en PDF deberá contener marcadores con la misma estructura y orden que el índice del Trabajo, al que se antepondrá la portada. Para facilitar la construcción de los marcadores, se recomienda construir el documento PDF desde la herramienta Crear PDF de Word, que aparece al instalar la Macro Acrobat en Microsoft Word. \\

En el directorio raíz del CD/DVD debe encontrarse al menos un único fichero: APELLIDO1\_APELLIDO2\_NOMBREDELALUMNO\_TFG.pdf que debe ser fiel imagen electrónica del documento que habría resultado si el trabajo se hubiera presentado impreso en papel.\\

El tamaño de las hojas incluidas en el fichero debe ser A4. \\

El CD/DVD debe ir contenido en una caja con tapa transparente (que permita ver la carátula del CD/DVD contenido) de dimensiones aproximadas 140 x 125 x 10 mm.\\

Con objeto de conseguir la uniformidad de los textos presentados, se establecen las indicaciones relacionadas a continuación.\\

%---------------------------------------
\subsection{Configuración de página} \label{sec:configuracion}
%---------------------------------------
El tamaño de las páginas será A4 con márgenes superior e inferior de 2,5 cm y derecho e izquierdo de 2 cm. Este tamaño podrá variarse cuando el contenido de la página así lo exija (por ejemplo, en planos, tablas resumen, figuras significativas) utilizando otros formatos.\\

No existirá encabezado ni pie para la portada y contraportada, y para las demás (igual para las páginas pares e impares) se incluirá en el encabezado el título del trabajo y el del capítulo y en el pie el número de página centrado. \\

La numeración del índice se realizará con número romanos. La numeración del documento se realizará con números arábicos. La numeración de los planos será independiente.\\

%---------------------------------------
\subsection{Estilos de texto} \label{sec:estilo}
%---------------------------------------
Los estilos de texto definen el tipo y tamaño de letra, la alineación, sangría y espaciado del texto a utilizar en los títulos de los capítulos, apartados, subapartados, pies de figuras y títulos de tablas. Los estilos a utilizar se resumen en la Tabla \ref{Tab:Tabla_Estilo}.

\begin{table}[h!]
\centering
\caption{Resumen de estilos definidos en el documento}
\resizebox{17cm}{!}{
\begin{tabular}{ | c | c | c | c | c | c | c | c | c | c | } 
\hline
\cellcolor{lightgray} \textbf{Estilo}  & \cellcolor{lightgray} \textbf{Tipo de letra} & \cellcolor{lightgray} \textbf{Tamaño} & \cellcolor{lightgray}  \textbf{Tipo} & \cellcolor{lightgray} \textbf{Alineación} &  \cellcolor{lightgray} \textbf{Sangría Izquierda} &  \cellcolor{lightgray} \textbf{Sangría Derecha} &  \cellcolor{lightgray} \textbf{Espaciado Anterior} & \cellcolor{lightgray}  \textbf{Espaciado Posterior} & \cellcolor{lightgray} \textbf{Interlineado}\\

\hline
Normal & Calibrí & 12 & Normal & Justificada & 0 & 0 & 10 & 0 & Sencillo\\
\hline

Título & Calibrí & 24 & Negrita & Centrada & 0 & 0 & 12 & 3 & Sencillo\\
\hline

Título 1 & Calibrí & 16 & Negrita mayúscula & Izquierda & 0 & 0 & 12 & 3 & Sencillo\\

\hline

Título 2 & Calibrí & 14 & Negrita & Izquierda & 0 & 0 & 12 & 3 & Sencillo\\
\hline

Título 3 & Calibrí & 13 & Negrita & Izquierda & 0 & 0 & 12 & 3 & Sencillo\\
\hline

Epígrafe &  Calibrí & 10 & Negrita & Centrada & 0 & 0 & 18 & 6 & Sencillo\\
\hline

Encabezado & Calibrí & 10 & Mayúscula & Título: Derecha. Capitulo: Izquierda. & 0 & 0 & 18 & 6 & Sencillo\\
\hline

Pie de página & Calibrí & 11 & Normal & Centrado & 0 & 0 & 0 & 0 & Sencillo\\
\hline

Nota al pie de página & Calibrí & 10 & Normal & Justificada & 0 & 0 & 0 & 0 & Sencillo\\
\hline
\end{tabular}
\label{Tab:Tabla_Estilo}
}
\end{table}

%---------------------------------------
\subsection{Listas numeradas y no numeradas}
%---------------------------------------
Las listas numeradas, conforme a la norma UNE 50132:1994, deben seguir el formato de ejemplo siguiente:

\begin{enumerate}[label*=\arabic*]
	\item Ejemplo de la justificación de los textos contenidos en las listas numeradas que será igual que para el caso de las no numeradas.
	\item
	\item
\end{enumerate}

Las listas no numeradas seguirán este otro:

\begin{itemize}
	\item Ejemplo de la justificación de los textos contenidos en las listas no numeradas que será igual que para el caso de las numeradas.
	\item
	\item
\end{itemize}

%---------------------------------------
\subsection{	Figuras, tablas y ecuaciones}
%---------------------------------------
Las tablas se alinearán horizontalmente en el centro de la página, presentarán un título encima de la misma que incluirá la palabra “Tabla”, el capítulo al que pertenece y un número secuencial dentro del capítulo. Siempre que aparezca una tabla debe existir en el texto una referencia a la misma. Ejemplo: la Tabla \ref{Tab:Tabla_Estilo}, donde se resumen los estilos de texto a utilizar \footnote{Nota al pie de página. Copiar SIEMPRE,  el comando después del texto de la nota de pie de página (Mirar el código)}.\\

Las figuras se alinearán conforme a su tamaño e inserción en el documento. Presentarán un título debajo de la figura, centrado sobre la misma, que incluirá “Figura”, el capítulo en el que se incluyen y un número secuencial dentro del capítulo. Siempre que aparezca una figura debe existir en el texto una referencia a ella. Ejemplo: en la Figura \ref{fig:Formato_Disco} se muestra el aspecto que presentará el CD/DVD que contendrá al TFG/M.\\
\newpage

\begin{figure}[h]
	\centering
	\includegraphics[width=0.65\textwidth]{Formato_Disco.png}
	\caption{Carátula del CD donde se presenta el PFC.}
	\label{fig:Formato_Disco}
\end{figure}  

Las ecuaciones se alinearán en el centro de la página y se numerarán a la derecha (se recomienda utilizar tabulaciones de alineación) indicando el capítulo y el número secuencial dentro del capítulo entre paréntesis, tal y como aparece en la ecuación (\ref{eq:eq_ejemplo}) de ejemplo siguiente:

\begin{eqnarray}
\label{eq:eq_ejemplo}
y & = & ax^2 + bx + c
\end{eqnarray}

El tamaño de letra de las ecuaciones debe ser el mismo que el del estilo Normal del documento (12 puntos).

