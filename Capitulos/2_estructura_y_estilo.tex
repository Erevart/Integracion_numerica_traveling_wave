%---------------------------------------
\section{ESTRUCTURAS Y ESTILOS DE REDACCIÓN} \label{sec:Estructuras}
%---------------------------------------
La estructura documental de un proyecto técnico de ingeniería consta, conforme a la norma UNE 157001:2002, de los siguientes documentos básicos: Memoria, Cálculos, Anexos, Planos, Pliego de Condiciones, Estado de Mediciones y Presupuesto. A estos documentos se les podrá añadir aquellos con entidad propia como el Estudio de Seguridad y Salud Laboral, etc.\\

En el caso particular en que los documentos a redactar correspondan a un anteproyecto, estudio, trabajo especial, trabajo de investigación y desarrollo o trabajo de otras características, el alumno puede simplificarlos tanto en su número como en su contenido, acoplándolos a las circunstancias de cada problema. En todo caso, será imprescindible que el proyecto quede definido en forma tal que otro facultativo con titulación suficiente pueda interpretar o dirigir con arreglo al mismo los trabajos correspondientes. En la redacción de un documento se hará referencia a cualquiera de los otros cuando así convenga para la interpretación completa del proyecto.\\

Como norma general de estilo se recomienda que la redacción de los títulos y de las oraciones sea directa y completa, los párrafos cortos y el estilo impersonal y objetivo (Por ejemplo: "han sido analizados” en lugar de: “analizamos”).

%---------------------------------------
\subsection{Capítulos, apartados y subapartados} \label{sec:Capitulos}
%---------------------------------------

En general, la memoria se dividirá en capítulos. El capítulo o división de mayor rango tendrá como numeración un solo número y siempre encabezará página. Ejemplo es el título que encabeza esta página: “\ref{sec:Estructuras} ESTRUCTURA Y ESTILO DE REDACCIÓN”.\\

La numeración del apartado estará integrada por el número de su correspondiente capítulo, seguido de un punto y otro número correlativo que partirá del 1, conforme a la norma UNE 50132:1994. Ejemplo es el presente apartado: “\ref{sec:Capitulos} Capítulos, apartados y subapartados”.

%---------------------------------------
\subsubsection{Subapartado} \label{sec:Subapartado}
%---------------------------------------
La numeración del sub-apartado estará integrada por el número de su correspondiente capítulo, seguido de punto, del número de su respectivo apartado, de otro punto, de otro número correlativo que partirá del 1.\\

No se recomienda dividir el trabajo en más de tres niveles, sin embargo si por alguna circunstancia interesara hacerlo, en las subdivisiones de nivel inferior se seguirían instrucciones análogas a las correspondientes al sub-apartado.

%---------------------------------------
\subsubsubsection{División menor} \label{sec:division menor}
%---------------------------------------

Se puede utilizar una división menor sin numeración, que por tanto no figurará en el índice, que consiste simplemente en un título, sin sangría, en negrita, escrito en minúsculas salvo su primera letra que será mayúscula.
