%---------------------------------------
\section{CONCLUSIONES} \label{sec:conclusiones}
%---------------------------------------
El presente informe pone de manifiesto una nueva expresión de movimiento de onda viajera, el cual permite regular el crecimiento y modulación de la amplitud de las oscilaciones con un único parámetro, frente a las expresiones clásicas que requieren la combinación de varios parámetros para alcanzar los mismos objetivos. Además presenta las funciones necesarias para calcular mediante integración numérica la velocidad de avance lograda por este nueva expresión y las ya conocidas.\\

Respecto a los resultados extraídos del presente documento, permiten afirmar que el movimiento de onda viajera fraccionaria es más óptimo desde el punto de vista de la fuerza de propulsión y con ello la velocidad de avance, respecto a las otros tipos de movimientos indicados. Pues como han demostrado los resultados, la onda viajera fraccionaria presenta un mayor crecimiento de las oscilaciones y al mismo tiempo realiza una mejor modulación de las mismas, lo cual permite alcanzar un 70\% de la amplitud final en tan solo un 20\% de la distancia a la que se logra la amplitud máxima, frente a un 21\% de una onda viajera con crecimiento lineal. Por otro lado, en la velocidad de avance se alcanza el 82\% de la velocidad ideal que se conseguiría con un movimiento armónico, en comparación al 47\% de la onda de crecimiento lineal. Así mismo, la comparación de los resultados proporcionados por la onda viajera fraccionaria respecto a los obtenidos por la onda de crecimiento modulado, permite afirmar que dicho movimiento describe un movimiento mas óptimo y eficiente, ya que esta última únicamente lograr incrementar varias unidades los rendimientos correspondientes a la onda viajera de crecimiento lineal.\\

Finalmente, es posible afirmar que la expresión de onda viajera fraccionaria indicada ofrece un movimiento de propulsión más óptimo, lo cual se traduce a su vez una mayor fuerza de propulsión y velocidad de avance.