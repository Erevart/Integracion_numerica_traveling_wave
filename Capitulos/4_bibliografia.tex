%---------------------------------------
\section{BIBLIOGRAFÍA} \label{sec:bibliografia}
%---------------------------------------
El documento incluirá un apartado (normalmente el último) bien con el título BIBLIOGRAFÍA, o bien REFERENCIAS, o bien BIBLIOGRAFÍA Y REFERENCIAS, en el que se relacionarán de forma numerada los distintos documentos (libros, artículos, normas, reglamentos,…) consultados o de aplicación al trabajo realizado.\\

En el caso de proyectos técnicos este apartado se denominará NORMAS Y REFERENCIAS y tendrá los subapartados que marca la norma UNE-EN 157001.\\

Se recomienda el empleo de numeración arábiga entre paréntesis cuadrados (por ejemplo [17]) dentro del texto para indicar la referencia, y escribir la lista de referencias al final del texto empleando espaciado simple como en el resto del trabajo.\\

La lista de referencias bibliográficas se ordena siguiendo el orden de aparición en el texto y por tanto la numeración asignada.
El formato adecuado para los diferentes tipos de referencias es el siguiente:
\begin{itemize}
\item \textbf{Artículo de revista}. Autor(es) (Inicial del nombre seguido de un punto y apellido), título del artículo (entre comillas), nombre de la revista (en itálica), volumen nº, fecha publicación, páginas (“pp” inicial-final).
\item \textbf{Anales (Proceedings) de Congresos}. Similar a artículo de revista.
\item \textbf{Libro}. Autor(es) (Inicial del nombre seguido de un punto y apellido), título del libro (en itálica) y edición, editorial, año.
\item \textbf{Leyes y reglamentos}. Título de la ley o reglamento tal y como aparece publicado oficialmente (en itálica), documento legal donde se publicó, número de documento y año de publicación.
\item \textbf{Normas}. Código con año de publicación y nombre de la norma (en itálica).
\item \textbf{Página web}. Nombre de página y/o documento consultado,\\
\ <dirección de internet>,\\
\ Consultada el día…
\item \textbf{Otros}: tesis, patentes, etc.
\end{itemize}

Se incluye en la bibliografía de este documento una lista de referencias como ejemplo. 


