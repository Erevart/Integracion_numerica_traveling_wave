%---------------------------------------
\section{CÁLCULO DE LA VELOCIDAD PARA ONDAS PEQUEÑAS} \label{anex:Anexo1}
%---------------------------------------
A continuación se muestra la función desarrollada para el cálculo de la velocidad de avance cuando se supone que las oscilaciones son pequeñas y la longitud de onda es similar a la longitud el flagelo.

\begin{lstlisting}[]
function [ NVx, TVx, t] = vxs(c0,c1,c2,alfa,teorico,Tf,Ts)
%VXS Calcula la velocidad de avance generada por una onda viajera del
%   tipo y = (c0 + c1*x^alfa + c2*x^2) * sin((2*pi/lambda)*(x-Vp*t)), bajo
%   las condiciones de oscilaciones de pequeña señal.
%
%   [NVx, TVx, T] = vxs(c0,c1,c2,alfa), NVx, es la velocidad generada por el
%   movimiento anterior con los coeficientes indicados, calculada mediante
%   integración numérica. TVx es la velocidad de propulsión calculada de
%   mediante la expresión algebraica, únicamente es calculada cuando el 
%   parametro teorico es verdadero. T denota el vector de tiempo.
%
%   'Teorico', valor boleano que permite calcular la velocidad mediante su
%   expresión algebraica. Únicamente válido para alfa = 1.
%
%   'Tf', indica el tiempo máximo de vector de tiempo T, que será devuelto.
%
%   'Ts', indica el paso de muestro del vector de tiempo.
%
%   Ejemplo:
%       Calcular el valor numérico y téorico.
%       [ NVC0, TVC0,T] = vxs(4e-6,0,0,1,1);
%
%   Ejemplo:
%       Calcular el valor numérico.
%       [ NVC1, TVC1,T] = vxs(0,4/24,0,1,0);
%
%   Ejemplo:
%       Calcular velocidad onda viajera fraccionaria.
%       [ NVF0,~,T] = vxs(0,4e-6/24e-6^.2,0,0.2,0);
%
%   Ejemplo:
%       Calcular velocidad para diferentes tiempos.
%       [ NVF0,~,T] = vxs(0,4/3*4/24,-4e-6/(3*24e-6^2),1,0,5e-2,1e-5);
%
%   
%   See also QUADGK.

    % -----------------------------
    % Parametros de Simulación
    % -----------------------------
    if (nargin < 5)
        Ts = 1e-5;      % Tiempo de Muestro
        Tf = 5e-2;      % Tiempo de Simulación
        teorico = 0;
    elseif (nargin < 6)
        Ts = 1e-5;      % Tiempo de Muestro
        Tf = 5e-2;      % Tiempo de Simulación    
    elseif (nargin < 7)
        Ts = 1e-5;      % Tiempo de Muestro    
    end

    t = 0:Ts:Tf;        % Vector de tiempo.
    NVx = [];

    % -----------------------------
    % Valores de la ecuación diferencial
    % -----------------------------

    mu = 7e-4;        % Viscosidad del medio. Fuente: Simulation of Swimming
                      %     Nanorobots in Biological Fluids (ieenano)
    f=35;             % Frecuencia de propagación.
    lambda = 24e-6;   % Longitud de onda de la onda viajera.
    n = 3;            % Numero de ondas simultaneas en el flagelo.      
    Vp = lambda * f;  % Velocidad de propagación de la onda viajera.
    R = 0.5e-6;       % Radio de la cabeza 
    d = 0.2e-6;       % Radio del flagelo
    L = n*lambda;     % Longitud del flagelo

    % Se puede calcular mirar ABF 
    % THE PROPULSION OF SEA-URCHIN SPERMATOZOA BY J. GRAY* AND G. J. 
    CL = - (2*pi*mu)/(log(d/(2*lambda))+(1/2));  
    CN = 2*CL;

    Cc = 6*pi*mu*R;          % Coeficiente de arrastre de la cabeza

    %% Calculo de la velocidad de avance por integración teórica.
    % Solo válida para teorico = 1

    if (teorico == 1)
        
        TVx = pi * f * ((CN - CL)/CL) *...
            ( 1/ (1 + (6 * pi * mu * R)/(n * lambda * CL) ) )...
            * ( -2*pi*c0^2/lambda - 2*pi*c0*c1 - (4*pi*lambda*c0*c2)/3 ...
            + c0*c1*sin(4*pi*f*t) + c0*c2*sin(4*pi*f*t)...
            - 2*pi*lambda*c1^2/3 - pi*lambda^2*c1*c2...
            + (lambda*c1^2/2)*sin(4*pi*f*t) + (lambda^2)*c1*c2*sin(4*pi*f*t)...
            - (2*pi*lambda^3*c2^2)/5 + (lambda^3*c2^2/2)*sin(4*pi*f*t) );
    else
        TVx = 0;
    end

    %% Calculo de la velocidad de avance por integración numérica.

    % Vx = (n * (CN - CL)) / (Cc + n * lambda * CL) * ( dy / dt * dy/dx dx)
    % Vx = |------------- vx_coeff ---------------| * |------ vx_dx -------|
    %                                         |-- dy_dt * (dy_dx1 + dy_dx2) --|

    vx_coeff = (  (1/lambda) * (CN - CL) / CL )...
                * ( 1 / (1 + Cc/ ( n * lambda * CL) ) ) ;

    for i=1:length(t);

        dy_dt =@(x) -(2*pi*Vp/lambda) * (c0 + c1.*x.^alfa  + c2.*x.^2)...
                    .* cos( (2*pi/lambda) * (x - Vp*t(i)) );

        dy_dx1 =@(x) (2*pi/lambda) .* (c0 + c1.*x.^alfa + c2.*x.^2)...
                    .* cos( (2*pi/lambda) * (x - Vp.*t(i)) );

        dy_dx2 =@(x) (2*c2.*x + alfa.*c1.*x.^(alfa-1))...
                    .* sin( (2*pi/lambda) * (x - Vp.*t(i)) ); 

        vx_dx =@(x) dy_dt(x) .* (dy_dx1(x) + dy_dx2(x));                                     

        % Función de integración numérica -- quadgk
        % Numerically evaluate integral, adaptive Gauss-Kronrod quadrature.
        NVx = [NVx  quadgk(vx_dx,0,lambda)];
    end

    NVx = NVx .* vx_coeff;
    
end
\end{lstlisting}
