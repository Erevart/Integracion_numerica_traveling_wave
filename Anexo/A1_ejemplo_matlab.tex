%---------------------------------------
\section{EJEMPLO DE CÓDIGO DE MATLAB} \label{anex:Anexo1}
%---------------------------------------
Estos es un Anexo, donde se muestra un ejemplo de como escribir un código de Matlab. La actual versión de la plantilla no admite tildes en el código.

\begin{lstlisting}[]
% Parametros de funcion de transferencia tipo
wn = sqrt(m3*m1);
d = (m3+m1)/(2*wn);
% Ganancia
k = -m2/(m3*m1);
% Polos
p1 = -d*wn+i*wn*sqrt(1-d^2);
p2 = -d*wn-i*wn*sqrt(1-d^2);

% Simulacion
y = lsim(G,u,t);
y0 = lsim(G0,u,t);
plot(t,y,t,y0,'--','linewidth',2);
legend('SS','FdT','Orginal')
set(gca, 'fontsize', 44);
grid on
ylabel('Nivel de glucosa','Interpreter','latex','FontSize',46)
xlabel('Tiempo (s)','Interpreter','latex','FontSize',46)
title('Metabolismo de la glucosa','Interpreter','latex','FontSize',46)
\end{lstlisting}
