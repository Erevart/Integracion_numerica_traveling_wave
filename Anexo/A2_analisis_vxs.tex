%---------------------------------------
\section{REPRESENTACIÓN DE RESULTADOS DE ONDA PEQUEÑA} \label{anex:Anexo2}
%---------------------------------------
A continuación se muestra el script desarrollado para la representación y cálculo de la velocidad de avance cuando se supone que las oscilaciones son pequeñas y los términos de segundo orden pueden ser obviados.

\begin{lstlisting}[]
%% VERIFICACIÓN Y ANÁLISIS VXS
% Este script realiza la representación de diferentes tipos de ondas
% viajeras, y calcula la velocidad de avance generada por cada una de
% ellas. Realizando posteriormente una gráfica de comparativa de
% velocidades.
% Cálculos realizados en este script atiendes nas consideraciones de onda
% de pequeña amplitud.

%% Representación movimiento de onda viajera clasico
figure(1)
print_traveling_wave(4e-6,0,0,1,[  0    0.4470    0.7410]);
hold on

print_traveling_wave(0,4/24,0,1,[0.4660    0.6740    0.1880]);

% Para C1 y C2, n y m son multiplos de la longitud de onda.  

% C1 = (A / (n*lambda) ) + ( m / (m - n) ) 
%    * (A / (m * lambda) - n * A / (m * lambda) )

% C2 = ( m / (m - n) ) * (A / ( n * m * lambda^2) - A / (m * lambda^2) )

% Para n = 1 y m = 3
print_traveling_wave(0,4/3*4/24,-4e-6/(3*24e-6^2),1,[0.6350,0.0780,0.1840]);

% Para n = 1 y m = 2
%print_traveling_wave(0,3/2*4/24,-4e-6/(2*24e-6^2),1,[0.6350,0.0780,0.1840]);
hold off

%% Representación movimiento de onda viajera fraccionario
figure(2)
print_traveling_wave(4e-6,0,0,1,[  0    0.4470    0.7410]);
hold on
print_traveling_wave(0,4e-6/24e-6^.2,0,0.2,[0.8500    0.3250    0.0980]);
print_traveling_wave(0,4e-6/24e-6^.4,0,0.4,[0.9290    0.6940    0.1250]);
print_traveling_wave(0,4e-6/24e-6^.6,0,0.6,[0.4940    0.1840    0.5560]);
print_traveling_wave(0,4/24,0,1,[0.4660    0.6740    0.1880]);
hold off


%% Verificación y validación del algoritmo de integración numérica.
clear all;
% Vector de tiempo
t = [];

% Cálculo velocidad de avance
[ NVC0, TVC0,t] = vxs(4e-6,0,0,1,1);
[ NVC1, TVC1,~] = vxs(0,4/24,0,1,1);
[ NVC2, TVC2,t] = vxs(0,4/3*4/24,-4e-6/(3*24e-6^2),1,1);
[ NVF0,~,~] = vxs(0,4e-6/24e-6^.2,0,0.2,0);
[ NVF1,~,~] = vxs(0,4e-6/24e-6^.4,0,0.4,0);
[ NVF2,~,~] = vxs(0,4e-6/24e-6^.6,0,0.6,0);


%% Representación de los resultados de velocidad
% Comparación de velocidades para C0
figure(3)
subplot(211)
plot(t,TVC0.*1e6,'Linewidth',4)
hold on
plot(t,NVC0.*1e6,'--r','Linewidth',4);
hold off
grid
ylabel('Velocidad ($\mu$m/s)','Interpreter','latex','FontSize',40)
title('Velocidad de avance para c$_0$','Interpreter','latex','FontSize',44)
tiy=get(gca,'ytick')';
set(gca, 'fontsize', 44,'yticklabel',num2str(tiy,'%.2f'));
legend('Valor Teórico','Valor Numérico');

subplot(212)
% RMSE
plot(t,sqrt(((NVC0 - TVC0).^2)./length(TVC0)).*1e2,'Linewidth',1)
grid
ylabel('RMSE (\%)','Interpreter','latex','FontSize',40)
xlabel('Tiempo (s)','Interpreter','latex','FontSize',40)
set(gca, 'fontsize', 44);

% Comparación de velocidades para C1
figure(4)
subplot(211)
plot(t,TVC1.*1e6,'Linewidth',4)
hold on
plot(t,NVC1.*1e6,'--r','Linewidth',4);
hold off
grid
ylabel('Velocidad ($\mu$m/s)','Interpreter','latex','FontSize',40)
title('Velocidad de avance para c$_1$','Interpreter','latex','FontSize',44)
set(gca, 'fontsize', 44);
legend('Valor Teórico','Valor Numérico');

subplot(212)
% RMSE
plot(t,sqrt(((NVC1 - TVC1).^2)./length(TVC1)).*1e2,'Linewidth',1)
grid
ylabel('RMSE (\%)','Interpreter','latex','FontSize',40)
xlabel('Tiempo (s)','Interpreter','latex','FontSize',40)
set(gca, 'fontsize', 44);


% Comparación de velocidades para C2
figure(5)
subplot(211)
plot(t,TVC2.*1e6,'Linewidth',4)
hold on
plot(t,NVC2.*1e6,'--r','Linewidth',4);
hold off
grid
ylabel('Velocidad ($\mu$m/s)','Interpreter','latex','FontSize',40)
title('Velocidad de avance para c$_1$ + c$_2$',...
        'Interpreter','latex','FontSize',44)
set(gca, 'fontsize', 44);
legend('Valor Teórico','Valor Numérico');

subplot(212)
% RMSE
plot(t,sqrt(((NVC2 - TVC2).^2)./length(TVC2)).*1e2)
grid
ylabel('RMSE (\%)','Interpreter','latex','FontSize',40)
xlabel('Tiempo (s)','Interpreter','latex','FontSize',40)
set(gca, 'fontsize', 44);


%% Representación de ñps resultados de velocidad de onda fraccionaria
figure(6)
plot(t,NVC0.*1e6,'Linewidth',4)
hold on
plot(t,NVF0.*1e6,'Linewidth',4);
plot(t,NVF1.*1e6,'Linewidth',4);
plot(t,NVF2.*1e6,'Linewidth',4);
plot(t,NVC1.*1e6,'Linewidth',4);
hold off
grid
ylabel('Velocidad ($\mu$m/s)','Interpreter','latex','FontSize',40)
xlabel('Tiempo (s)','Interpreter','latex','FontSize',40)
title('Velocidad onda viajera fraccionaria',...
        'Interpreter','latex','FontSize',44)
set(gca, 'fontsize', 44);
legend('Onda armónica','\alpha = 0.2','\alpha = 0.4',...
        '\alpha = 0.6','\alpha = 1');
\end{lstlisting}
